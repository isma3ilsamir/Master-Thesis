\chapter{Conclusion and Future Work}
\label{ChapterConclusion}
The primary goal of this thesis is to compare the performance of TSCAs in an early classification context.
A context which is inpsired by the problem of eTSC. Our research questions investigated adapting workflows
for the early classifiation context, how to evaluate the classifiers in the context while including earliness and performance
and how to evaluate the proposed solution.

We introduced a new framework that simulates the early classification context and compares TSCAs on the UCR/UEA data archive using a testbed.
The framework is then evaluated by building a recommender which learns on the results and predicts for unseen data sets which are the best classifiers to use.

Our proposed implementation of the context cannot be handled by all classifiers.
They either have to be able to compare instances of different lengths or involve a feature extraction step that creates a fixed feature vector size.
Classifiers that are bound by the shape of data fail to operate in the context.
Our final experiment results included the classifiers CBoss, PForest, ST, TSF and WEASEL compared on chosen data sets from the UCR/UEA time series data archives.
Along with a baseline classifier that predict majority class.

We use the $F_{\beta}-measure$ as an overall performance metric; as it combines both performance of classifier and earliness.
The value of $\beta$ plays a fundamental role in the results and should be adjusted according to the needs of the experiments.
Using equal weights for performance and earliness, overemphasizes the importance of earliness, which consequently overcompensates for the lower performace scores.

Overall, our testbed results for the across classifiers comparison show that TSF is, on average, better than the other 4 classifiers on early chunks of data.
It is clearly superior on the 10\% chunk data, while the others are no better than the majority class baseline.
As more data is revealed to learn on, the other classifiers; CBoss, WEASEL, ST and PForest, start to catch up with TSF.
On the full length data, the 5 classifiers are not significantly different from each other and are all better than the baseline, yet WEASEL has the highest rank.

For the within classifier comparison based on $F_{\beta}-measure$, our testbed results show that there is no statistical difference between the 10\% version of any of the classifiers and their respective 100\% versions.
This contrasts with the within comparison based on balanced accuracy, where the 100\% versions are significantly better than the 10\% versions for the classifiers.
This means that the earliness factor plays a big role in shaping the results; it gives an advantage to the lower chunks that compensates for the lower performance.
A research interest that we couldn't infer useful insights for is the effect of the different data characteristics on the performance of classsifiers across the different chunks.

To evaluate the created framework, we built a recommender that predicts the performance of TSCAs for unseen data sets in the early classification context.
The recommender is better in predicting the $F_{\beta}-measure$ of classifiers for data sets on the lower chunks scoring an average 6.71\% RMSE error,
but gets worse when predicting the scores on higher chunks till the RMSE reaches 19.52\%.
The train size, test size, length and number of classes play an important role in predicting the chunk learners results.
Other data set characteristics like; problem type and number of dimensions don't seem to have a preciptable relation with the results.

For every chunk, the recommender is good at judging whether a classifier is a good performer or not, despite the residuals between the actual and predicted $F_{\beta}-measure$ values.
This goes back to the central role that the baseline classifier plays in the recommendation part.
The baseline classifier is generally worse than all TSCAs on all chunks, so it is easy to beat on many data sets.
This causes our recall values to be very high, specially on the 100\% chunk data when the baseline classifier is significantly worse than all the others.

There are multiple weaknesses in our study that we freely acknowledge.
The selection criteria of the data set problem types is cherry picked to suit our motivating goal, the UCR/UEA data archives contains other problem types that we have not
included which cover a variety of other time series problems.
There are multiple components and parameters of the framework that have a great effect on the final results.
For example, the evaluation metric to use for comparing the classifiers.
We emphasize that $F_{\beta}-measure$ is not the only suitable measure for assessing classifiers performance.
But if used, the choice of weights has a great effect that should be considered in interpreting the results.
In handling multivariate data sets, our choice for the column ensembling technique by repeating the same classifier for each dimension
might not be the best way for evaluating the real performance of the classifiers on these data sets.
Our used configurations for the classifiers and the adaptations applied might not be fair for all of them.
For example, we use the contratable versions of CBoss and ST and used a 1 hour time contract.
This might have hindered their actual performance on the data sets and might have scored better if given more time.
Also the exclusion of the TWE elastic distance metric from the PForest classifier to meet time contraints,
might have introduced a bias in the results for PForest.

We believe there are a lot of areas that are open for more research and improvement in the early classification context.
Handling multivariate data sets is one of the areas.
For instance, the inclusion of bespoke multivariate classifiers to learn on multivariate data sets while capturing relations between dimensions.
Running experiments using multiple resampling of data instead of the default train/test split; to avoid uncertainity about the performance of classifiers on data sets.
Deeper and more granular analysis of the features used by the classifiers; either as shapelets for ST or temporal importance curves for TSF.
Using a different baseline in the recommender for example KNNDTW for deciding good and bad performing classifiers.