The Euclidean distance is a remarkably simple technique to calculate the distance between time series instances.
Given two instances  $T\textsubscript{1}$ = [$t1\textsubscript{1}$,$t1\textsubscript{2}$,...$t1\textsubscript{n}$]
and $T\textsubscript{2}$ = [$t2\textsubscript{1}$,$t2\textsubscript{2}$,...$t2\textsubscript{n}$], the euclidean distance
between them can be determined as:
\begin{definition}
    ED($T\textsubscript{1}$,$T\textsubscript{2}$)= $\sqrt{\sum_{i=1}^{n} (t1\textsubscript{i} - t2\textsubscript{i})^{2}}$
\end{definition}
Euclidean distnace has been preferred to other classifiers due to it's space and time efficiency, but it suffers from two main shortcomings \cite{baydogan2013bag, jeong2011weighted,kate2016using}.
The first one is that it cannot handle comparisons between time series of different lengths.
While the second one is it's sensitivity to minor discrepancies between time series; it would calculate large distance values for small shiftings or misalignments.
Although other metrics have been introduced to overcome the drawbacks of euclidean distance,
experimental proof showed that this is only the case for small datasets, but for larger datasets the accuracy of other elastic measures
converge with euclidean distance \cite{hills2014classification,ding2008querying,bagnall2012transformation}.