Phase dependent algorithms is a group of algorithms that extract temporal features from intervals of time series.
These temporal features help with the interpretibility of the model; as they give insights about the temporal characteristics of the data \cite{baydogan2016time},
unlike whole time series algorithms that base their decisions solely on the similarities between instances.
Another advantage of phase dependent algorithms is that they can also handle distortions and misalignments of time series data \cite{deng2013time}.\newline
According to \cite{bagnall2017great}, phase dependent algorithms are best used for problems where discrimanatory information from intervals exist,
this would be the case with long timer series instances and which might include areas of noise that can easily deceive classifiers.
Like the case with the SmallKitchenAppliances dataset, in which the usage of three classes; a kettle, a microwave and toaster is recorded every 2 minutes for one day.
Not only the pattern of usage is discrimanatory in such case, but also the time of usage \newline
Typically using interval features requires a two phase process; first by extracting the temporal features and then training a classifier using the extracted features \cite{deng2013time}.
There are n(n-1)/2 possible intervals,for a time series of length n\cite{bagnall2017great}.
There is also a wide variety of features, also called literals, to extract for each interval. These cover simple statistical measures as well as local and global temporal features \cite{santos2016literature,rodriguez2004support,deng2013time}.
This introduces one of the main challenges for phase dependent algorithms, that is which intervals to consider for the feature extraction step.
Which \cite{rodriguez2004support} proposed a solution for by only considering intervals with lengths equal to powers of two \cite{bagnall2017great}.
