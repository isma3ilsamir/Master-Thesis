\chapter{Adaptation Methodology}
\label{ChapterAdaptationMethodology}
THis chapter should describe the Adaptation Methodology

%
As motivated earlier, eTSC is a field which is concerned with classification of time series data with earliness and accuracy
as the main objectives. Dedicated eTSCAs focus on adapting the algorithms to learn class labels of the data as early as possible
while maintaining accuracy \cite{mori2017early}.
Our framework, on the other hand, investigated the adaptation of the context in which conventional TSCA operate to learn a specific
classification problem and the effect it had on the classifiers' performance.

We define our early classification context as

We implement the context using an algorithm which given a time series $T$ of some length $L$ and a number of splits $s$,
the algorithm would cut $T$ into smaller subsequences of nearly equal lengths. Where the length of each subsequence is equal to $\div{L}{s}$.
$s$ is a parameter that decides the ratio represented by each subsequence length to the total length of the time series $T$,
that is if $s$ = 10 then the length of each subsequence would be 10\% of the total length, if $s$ = 20 then each subsequence
represents 5\% of the total length and so on.

This cutting mechanism controls is what decides the notion of earliness in our calculations.
