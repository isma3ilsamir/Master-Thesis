\chapter{Classification Algorithms}
\label{ChapterClassificationAlgorithms}
This Chapter discusses the definitions and background of the topics mentioned in this thesis.
We discuss the nature of time series data, the two problems of time series classification (TSC) and early time series classification (eTSC) then present the different techniques encompassed by them.
We will discuss TSCAs in more details as they are the main focus of our framework, but a review of the problem of early classification and a review of it's algorithms is also included;
as it motivates the context in which the framework operates.

%%%%%%%%%%%%%%%%%%%%%%%%%%%%%%%%%%%%%%%%%%%%%%%%%%%%%%%%%%%%%%%%%%%%%%%%%%%%
%%%%%%%%%%%%%%%%%%%%%%%%%%%%%%%%%%%%%%%%%%%%%%%%%%%%%%%%%%%%%%%%%%%%%%%%%%%%
\section{Time Series Data}
\label{TimeSeriesData}

\subsection{Definitions}
\label{TSDefinitions}
A time series is a finite sequence of ordered observations, either based on time or another aspect \cite{abanda2019review,bagnall2017great}.
The existence of the time component makes time series an abundant form of data that covers various domains like; medicine, finance, engineering and biology \cite{lines2018time}.

A time series dataset is a collection of time series instances.
\begin{definition}
    \[ \textstyle \text{A set T of n time series instances, T = } \{T_{1}, T_{2}, \ldots ,T_{n}\}. \]
\end{definition}
Each of the time series instances $T_{i}$ consists of a sequence of observations collected at consequent time points.
\begin{definition}
    \[ \textstyle \text{A time series } T_{i} \text{, of length L is represented as } T_{i} = [t_{1}, t_{2}, \ldots t_{L}]. \]
\end{definition}
An observation $t_{i}$ represents the value of the collected measurement at some point of time $i$.

Time series data can come in different forms. It is important to comprehend what different forms the data can take and what implicit assumptions they convey; to be able to choose the suitable algorithms and tools to deal with it.

The first form of time series data is referred to as univariate time series.
This form of data exists when the observations of instances capture a singular value across time.
\begin{definition}
    \begin{align*} 
        & \text{Univariate time series } T_{i} \text{, of length L is represented as } T_{i} = [t_{1},t_{2}, \ldots t_{L}] \\
        & \text{ With } t_{j} \text{ as a real valued number.}
    \end{align*}
\end{definition}

The other form is when multiple measurments are captured by the observations.
According to \cite{loning2019sktime}, it is essential to differentiate between the two ways multiple time series can be generated; panel data and multivariate time series data.
If more than one variable is being observed during a single experiment, with each variable representing a different measurement; this is called multivariate time series.
\begin{definition}
    \begin{align*} 
        & \text{Multivariate time series } T_{i} \text{, of length L is represented as } T_{i} = [t_{1},t_{2}, \ldots t_{L}] \\
        & \text{ With } t_{j} \text{  having M dimensions, each one of them is a univariate time series.}
    \end{align*}
\end{definition}
While panel data is when the same kind of measurments is collected from independent instances; like different patients or diverse industrial processes.
For panel data, it is possible to assume that the different instances are independent and identically distributed (i.i.d), but this assumption doesn't hold for observation of a single instance.
The same goes for multivariate time series, individual univariate observations are assumed to be statistically dependant.

\subsection{Nature of Time Series Data}
\label{NatureOfTimeSeriesData}
Having discussed the dependency assumptions in time the different forms of time series data.
It is this dependency that makes time series data challenging for conventional machine learning algorithms, which are used for tabular and cross-sectional data.
Tabular and cross-sectional data assume observations to be i.i.d \cite{loning2019sktime}.

If we were to tabularize time series data; convert it into a tabular form by considering each observation as an individual feature.
Then it would be possible to apply conventional machine learning algorithms, under the implicit modelling assumption that observations are not ordered.
This means that if the order of the features was changed, still the model result will not change.
This assumption can work for some problems, but it doesn't have to work for all problems.

%%%%%%%%%%%%%%%%%%%%%%%%%%%%%%%%%%%%%%%%%%%%%%%%%%%%%%%%%%%%%%%%%%%%%%%%%%%%
%%%%%%%%%%%%%%%%%%%%%%%%%%%%%%%%%%%%%%%%%%%%%%%%%%%%%%%%%%%%%%%%%%%%%%%%%%%%
\section{Time Series Classification}
\label{TimeSeriesClassification}
Time series classification is a subtype of the general classification problem, which considers the the unique property of dependency between adjacent features of instances \cite{Bostrom2017}.
The main goal of time series classification is to learn a function $f$,
which given a training dataset $T = \{T_{1},T_{2}, \ldots ,T_{n}$\} of time series instances
along with their corresponding class labels $Y=\{y_{1},y_{2}, \ldots y_{n}$\} where $y_{i} \in \{1,2, \ldots C\}$,
can predict class labels for unseen instances \cite{deng2013time}.

Time series classification has been studied with different objectives, some papers focused on attaining the highest accuracy of classification as the main goal \cite{kate2016using,jeong2011weighted,bostrom2017shapelet,lines2018time,schafer2017multivariate,fawaz2020inceptiontime},
while other papers focused on attaining lower time complexity \cite{ratanamahatana2004making,bagnall2017great,tan2020fastee,petitjean2016faster,schafer2017fast}.

In this master thesis, we are more interested in assessing the results in terms of accuracy than time complexity. We define accuracy like \cite{schafer2020teaser}; 
as the percentage of correctly classified instances for a given dataset D, either being a training or testing dataset.
\begin{definition}
    \[ \textstyle Accuracy = \frac{ \text{number of correct classifications}}{|\text{D}|} \]
\end{definition}

%%%%%%%%%%%%%%%%%%%%%%%%%%%%%%%%%%%%%%%%%%%%%%%%%%%%%%%%%%%%%%%%%%%%%%%%%%%%
%%%%%%%%%%%%%%%%%%%%%%%%%%%%%%%%%%%%%%%%%%%%%%%%%%%%%%%%%%%%%%%%%%%%%%%%%%%%
\section{Time Series Classification Algorithms}
\label{SectionTSCA}
This chapter will introduce different types of TSCAs and discuss the various techniques that each type apply.
There are multiple ways to divide TSCAs in order to better understand them.
In this thesis we follow the grouping defined by \cite{bagnall2017great}.
\subimport{./}{Whole}
\subimport{./}{Phase Dependent}
\subimport{./}{Shapelet}
\subimport{./}{Dictionary}
\subimport{./}{Deep Learning}

%%%%%%%%%%%%%%%%%%%%%%%%%%%%%%%%%%%%%%%%%%%%%%%%%%%%%%%%%%%%%%%%%%%%%%%%%%%%
%%%%%%%%%%%%%%%%%%%%%%%%%%%%%%%%%%%%%%%%%%%%%%%%%%%%%%%%%%%%%%%%%%%%%%%%%%%%
\section{Early Time Series Classification}
\label{EarlyTimeSeriesClassification}
On another side, Early Time Series Classification (eTSC) is also a classification problem which considers the temporal nature of data,
but with a focus on slightly different objectives.  and had been used in multiple domains.

eTSC's main objective is to learn a model which can classify unseen instances as early as possible,
while maintaining a competitive accuracy compared to a model that uses full length data or to a user defined threshold\cite{xing2009early}.
Which is a very challenging objective; due to the, naturally, contradicting nature of earliness and accuracy.
In general, the more data is made available for the model to learn the better accuracy it can attain \cite{mori2019early,tavenard2016cost,xing2012early,mori2017reliable}.
This is why eTSC is considered as a problem of optimizing multiple objectives.

We have already defined accuracy for TSC in section \ref{TimeSeriesClassification}.
For earliness, we follow the definition mentioned by \cite{schafer2020teaser}; as the mean number of data points s after which a label is assigned.
\begin{definition}
    \label{DefinitionEarliness}
    \[ \textstyle Earliness = \frac{\sum_{T{i}\in D}\frac{s}{len(T_{i})}}{|D|} \]
\end{definition}
As well as the objective measure, Harmonic mean (HM), mentioned by \cite{ghalwash2012early,schafer2020teaser}, which combines both accuracy and earliness in one equation.
For the scope of our framework, we consider HM as a weighted average between accuracy and earliness.
\begin{definition}
    \label{DefinitionHM}
    $F_{\beta} = (1 + \beta^2)\frac{\text{accuracy(1 - earliness)}}{\beta^2 \text{(1 - accuracy) + earliness}}$
\end{definition}
The value of $\beta$ can be used to give higher importance to one of the aspects over the other, but we use equal weights for both.

eTSC is needed in situations in which waiting for more data to arrive can be costly or
when making late decisions can cause unfavorable results \cite{mori2017early,parrish2013classifying,lin2015reliable}.
This is why eTSC has been applied in various domains like early medical diagnosis \cite{griffin2001toward,ghalwash2012early},
avoiding issues in network traffic flow \cite{bernaille2006traffic}, human activity recognition \cite{yazdanbakhsh2019multivariate,gupta2020fault}
and early prediction of stock crisis \cite{ghalwash2014utilizing}.

\section{Early Time Series Classification Algorithms}
\label{SectionETSCA}
\subimport{./}{Early TSC Algorithms}