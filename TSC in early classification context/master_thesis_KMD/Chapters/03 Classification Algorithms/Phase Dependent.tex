\subsection{Phase Dependent intervals Algorithms}
\label{SubsectionPhaseDependent}
Phase dependent algorithms is a group of algorithms that extract temporal features from intervals of time series.
These temporal features help with the interpretibility of the model; as they give insights about the temporal characteristics of the data \cite{baydogan2016time},
unlike whole time series algorithms that base their decisions solely on the similarities between instances.
Another advantage of phase dependent algorithms is that they can also handle distortions and misalignments of time series data \cite{deng2013time}.

According to \cite{bagnall2017great}, phase dependent algorithms are best used for problems where discrimanatory information from intervals exist,
this would be the case with long timer series instances and which might include areas of noise that can easily deceive classifiers.
Like the case with the SmallKitchenAppliances dataset, in which the usage of three classes; a kettle, a microwave and toaster is recorded every 2 minutes for one day.
Not only the pattern of usage is discrimanatory in such case, but also the time of usage.

Typically using interval features requires a two phase process; first by extracting the temporal features and then training a classifier using the extracted features \cite{deng2013time}.
There are n(n-1)/2 possible intervals,for a time series of length n\cite{bagnall2017great}.
There is also a wide variety of features, also called literals, to extract for each interval. These cover simple statistical measures as well as local and global temporal features \cite{santos2016literature,rodriguez2004support,deng2013time}.
This introduces one of the main challenges for phase dependent algorithms, that is which intervals to consider for the feature extraction step.
Which \cite{rodriguez2004support} proposed a solution for by only considering intervals with lengths equal to powers of two \cite{bagnall2017great}.

\subsubsection{Time Series Forest}
\label{SubsubsectionTimeSeriesForest}
Time Series Forest (TSF) is an algorithm that was introduced in 2013 by \cite{deng2013time}.
They motivated their model with two main criteria; contributing to interpretable time series classification through
the use of simple statistical temporal features and reaching this goal by creating an efficient and effective classifier.

TSF considers three types of interval features; mean, standard deviation and slope.
If we were to consider an interval with starting point $t\textsubscript{1}$ and with ending point $t\textsubscript{2}$.
Let $v\textsubscript{i}$ be the value at a specific point $t\textsubscript{i}$.
Then the three features can be denoted as:
\begin{equation}
    mean(t\textsubscript{1},t\textsubscript{2})= \frac{\sum_{i=t\textsubscript{1}}^{t\textsubscript{2}} v\textsubscript{i}}{t\textsubscript{2} - t\textsubscript{1} + 1}
\end{equation}

\begin{equation}
    std(t\textsubscript{1},t\textsubscript{2})=
        \begin{cases}
        \frac{\sum_{i=t\textsubscript{1}}^{t\textsubscript{2}} (v\textsubscript{i} - mean(t\textsubscript{1},t\textsubscript{2}))^{2}}
            {t\textsubscript{2} - t\textsubscript{1}}
            & \text{if t\textsubscript{1} $<$ t\textsubscript{2}}\\
        0 & \text{if t\textsubscript{1} = t\textsubscript{2}}
        \end{cases}
\end{equation}

\begin{equation}
    slope(t\textsubscript{1},t\textsubscript{2})=
        \begin{cases}
        m & \text{if t\textsubscript{1} $<$ t\textsubscript{2}}\\
        0 & \text{if t\textsubscript{1} = t\textsubscript{2}}
        \end{cases}
\end{equation}
Where m denotes the slope of the least squares regression line for the training dataset.

For building the trees, TSF introduced a new splitting criteria at the tree nodes, which they called the Entrace. A combination of
Entropy and distance; to break the ties between features of equal entropy gain by preferring splits that have the furthest distance
to the nearest instance. They also use a specific number of evaluation points rather than checking all split points for the highest
information gain. In their experiment \cite{bagnall2017great} found these two criteria to have negative effect on accuracy.

As mentioned earlier the feature space for creating interval features is huge, TSF adopts the same random sampling technique that Random Forests use
reducing the feature space from $O(M^{2})$ to only $O(M)$, by considering only $O(\sqrt{M})$ random interval sizes and $O(\sqrt{M})$
random starting points at each tree node \cite{deng2013time}. The final classification of a testing instance is done using majority
voting of all time series trees created.