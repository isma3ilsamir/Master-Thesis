% !TEX root = main.tex

\setcounter{page}{1}
\section*{Abstract}
\thispagestyle{empty}
Time series classification (TSC) is a field of data mining that aims at learning predictive models that can assign class labels for time series.
Whether the time series is univariate or multivariate, time series data exists in many real life applications around us.
The abundance of time series data lead to an interest in extending TSC to incorporate early decision making.
A specialized field in learning early classification algorithms for time series, called Early Time Series Classification (eTSC), emerged.
This new field helps in scenarios where time is an important factor; like diseases diagnosis and natural disasters prediction.
Inspired by the eTSC problem and the competent Time Series Classification Algorithms (TSCAs), we define a context that simulates incremental learning of time series data and study its effect on the performance of TSCAs.
We review existing algorithms from both TSC and eTSC, as well as previous frameworks for comparing TSCAs.
Then implement a framework that simulates our proposed context through chopping of training data.
We conduct experiments on 77 data sets from the UCR and UEA data archives and make comparisons within and across different TSCAs.
The results demonstrate that each TSCA can achieve competent performance in the early classifiation context compared to its performance on the full length data.
When compared against other on full data length, the TSCAs are at their full capacity and so they all attain high performance scores on the data sets.
There is no statistical difference between any of the classifiers at that stage.
On the other hand, when compared on early chunks of data, their performances are affected by the data chopping and statistical differences exist between their performances.
TSF ranks as the best performing classifier on early chunk data compared to the others.
To evaluate our framework, we show that it is possible to learn a recommender to predict the performance of the TSCAs in the early classification context for unseen data sets
and suggest the good performing ones.

\null\newpage

\section*{Acknowledgement}
\thispagestyle{empty}
This master thesis has been a journey more than an academic work. Working during the COVID pandemic was not only a challenge but a whole new experience.
I would like to express my gratitude for my supervisor, Prof. Spiliopoulou, who introduced me to this interesting topic and who provided guidance through out the work
regardless of me living in another city and of the pandemic situation. Such a work would have not been possible, if it weren't for her understanding and tolerence.
I would also like to give special thanks to Noor Jamaludeen and Vishnu Unikrishnan for their constant support regardless of their busy schedules.
Their side discussions were an eye openner for me on a lot of topics and always guided me to the right path.
Likewise, I would like to thank my cousin Karim for helping me out and proof reading this work.
I would like to thank my friends Nazem, Ali Hashaam, Hossam, Moaaz and Nashaat for being there for me.
This thesis is specially dedicated to my family: my parents whom I wish to make proud and my brothers.
Moreover, I would like to thank my 9 months old daughter Hana who unintentionally made me realize that I should reprioritize my life and taught me to let go.
In the end, I would like to thank my wife Amira who has been an amazing partner through out this whole journey and pushed me to be the best version of myself.


\null\newpage
\thispagestyle{empty}


