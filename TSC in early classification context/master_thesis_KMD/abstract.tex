% !TEX root = main.tex

\setcounter{page}{1}
\section*{Abstract}
\thispagestyle{empty}
Time series classification (TSC) is a field of machine learning that aims at learning predictive models that can assign class labels for time series.
Whether the time series is univariate or multivariate, time series data exists in many real life applications around us.
The abundance of time series data lead to an interest in extending TSC to incorporate early decision making.
A specialized field in learning early classification algorithms for time series, called Early Time Series Classification (eTSC), emerged.
This new field helps in scenarios where time is an important factor; like diseases diagnosis and natural disasters prediction.
Inspired by the eTSC problem and the competent Time Series Classification Algorithms (TSCAs), we define a context that simulates incremental learning of time series data and study its effect on the performance of TSCAs.
We review existing algorithms from both TSC and eTSC, as well as previous frameworks for comparing TSCAs.
Then implement a framework that simulates our proposed context through chopping of training data.
We conduct experiments on 77 data sets from the UCR and UEA data archives and make comparisons within and across different TSCAs.
The results demonstrate that each TSCA can achieve competent performance in the early classifiation context compared to its performance on the full length data.
When compared against other on full data length, the TSCAs are at their full capacity and so they all attain high performance scores on the data sets.
There is no statistical difference between any of the classifiers at that stage.
On the other hand, when compared on early chunks of data, their performances are affected by the data chopping and statistical differences exist between their performances.
TSF ranks as the best performing classifier on early chunk data compared to the others.
To evaluate our framework, we show that it is possible to learn a recommender to predict the performance of the TSCAs in the early classification context for unseen data sets
and suggest the good performing ones.

\null\newpage

\section*{Acknowledgement}
\thispagestyle{empty}
Lorem ipsum dolor sit amet, consectetur adipiscing elit. Curabitur eget porta erat. Morbi consectetur est vel gravida pretium. Suspendisse ut dui eu ante cursus gravida non sed sem. Nullam sapien tellus, commodo id velit id, eleifend volutpat quam. Phasellus mauris velit, dapibus finibus elementum vel, pulvinar non tellus. Nunc pellentesque pretium diam, quis maximus dolor faucibus id. Nunc convallis sodales ante, ut ullamcorper est egestas vitae. Nam sit amet enim ultrices, ultrices elit pulvinar, volutpat risus.



\null\newpage
\thispagestyle{empty}


